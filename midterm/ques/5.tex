\section{}
\setcounter{equation}{0}
\subsection{}
Probability to solve hash-puzzle  $ = p$ \\ 
Ways to choose $k$ miners out of $n$ = $ \binom{n}{k}$ \\
For k miners, prob. to solve hash-puzzle simultaneously 
\begin{align}
	=& \binom{n}{k} (p)^k (1-p)^{n-k}
\end{align}
\subsection{}
Suppose in a certain time interval (t), we make $n$ tries to find a block and out of that only $k$ successes are achieved. The Probability to find a block (i.e. to solve a hash-puzzle) is $p$, and is fixed because the blockchain network changes the difficulty so that this remains constant.
\\ \\
Hence, in that time interval, the total number of blocks found out is equal to $\lambda = np$. Now, the distribution of finding a block in a large enough fixed interval follows the Binomial Distribution i.e. (Here, $N_t$ is the number of arrivals in a time interval t)
\begin{align}
	P(N_t = k \mid p, n) = & \binom{n}{k} (p)^k (1-p)^{n-k}
\end{align}
For a blockchain, the number $n$ is sufficiently high, so that we can take  \\
$n \to \infty$ and that $\lambda = np$, so that out $p \to 0$.
\begin{align}
	\lim_{n \to \infty} P(N_t = k \mid p, n) 
	&= \lim_{n \to \infty} \binom{n}{k} (p)^k (1-p)^{n-k} \\
	&= \lim_{n \to \infty} \binom{n}{k} \left(\frac{\lambda}{n}\right)^k 
		\left(1-\frac{\lambda}{n}\right)^{n-k} \\
	&= \frac{\lambda^k e^{-\lambda}}{k !}
\end{align}
This is the pmf for Poisson Distribution. \\ \\
Now consider, \\
	$X_t: $ The time it takes for mining one additional block assuming that some block was mined successfully at time $t$.
	By definition of $N_t$, we have
\begin{align}
	(X_t > x) \equiv (N_t = N_{t+x})
\end{align}
This means that the number of blocks which have arrived upto time $t$ to $t+x$ is equal to $N_t$, or that no new block has been mined between time interval $[t, t+x]$. \\
Now,
\begin{align*}
	P(X_t \leq x) 
	&= 1 - P(X_t > x)  \\
	&= 1 - P(N_{t+x} = N_{t}) \\	
	&= 1 - P(N_{t+x} - N_{t} = 0) \\	
	&= 1 - P(N_{x} = 0)	
\end{align}
The last step takes into fact that no new block was mined in time interval [t, t+x], and hence $N_x = 0$.  \\ \\
Also, $\lambda$ in Poisson Distribution is the expected number of total arrivals. Take $\alpha$ to be the average number of arrivals per unit time. Hence, for a time interval $x$, we have a total of $\alpha x$ arrivals.
From (5), we have, 
\begin{align*}
	P(X_t \leq x) 
	&= 1 - 	\frac{(\alpha x)^0 e^{-(\alpha x)}}{0!} \\
	&= 1 - 	e^{-(\alpha x)} \\
	X &\sim \exp({x; \alpha})
\end{align}
This is the CDF of exponential distribution, with an average time to mine a block of $\alpha$. Hence, the time taken to mine a block in the next $x$ time units has exponenttial distribution. \\
.\hfill{} \bo{[Q.E.D]}
\subsection{}
No, it's not possible to rely upon the timestamp provided by a miner to prioritize his block over someone else's. Due to network delays and congestion, a large amount of delay can generate between the time when a block was generated to the time when it was broadcasted.
As a result, we can't have time synchronized over the world.


\subsection{}
From part (b), the number of blocks which are mined in a certain time t follows exponential distribution.
The number of blocks expected to be found in a certain time follows the Poisson distribution.


\begin{align*}
	P(N_t \geq 6)  &= 0.99 \\
	&= 1 - P(N_t < 6) \\
	0.99 &= 1 - \sum_{i=0}^{5}P(N_t = i)
\end{align}
Now, the parameter $\lambda$ for Poission Distribution is the mean number of blocks expected blocks. For time $x$, we have,

\begin{align*}
	\sum_{i=0}^{5}P(N_t = i) &=1 - 0.99  \\
	\sum_{i=0}^{5} \frac{ (\alpha x)^{i} e^{-\alpha x} }{i!} &= 0.01
\end{align}
On solving, we get the value of $x \approx 130$ minutes
