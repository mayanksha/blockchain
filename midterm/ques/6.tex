\section{BoyCott possible?}

Bitcoin has no mechanism to penalize any malicious user, thereby a scheme wherein we BoyCott some node is not possible. Even if such a scheme were possible, the malicious person can simply create a new wallet address by generating a new public/private key pair and using that to mine blocks.
Blockchain also doesn't store any information related to the IP of miners hence an IP ban is out of the way too. Hence, boycotting any miner's chain can't prevent them from creating a new identity (priv/pub key pair) and start performing malicious behaviour again. \\ \\ 
Instead, what blockchain relies upon for promoting normal behaviour is the game-theoretic possibilities which a node can use to maximize his profits. Each node picks a strategy  to maximize its payoff by taking into account the other nodes' potential strategies. Every node is assumed to act according to its incentives, we can't split nodes into malicious or benign.
\\ \\ 

\section{Mining Pools}
The designer can change the hash puzzle problem from "Find the nonce such that the Hash belongs to the target space " to the new hash puzzle "Find the nonce and such that the hash of the digital signature belongs to a certain target space".  \\  \\
This setting automatically highly discourages a minig pool setup because calculating the digital signature requires the knowledge of private key of pool manager to all the nodes participating in the pool. 
\\ \\
Any malicious person can steal the private key and use this to steal the money from the bitcoin wallet to which a block reward payment goes to on successful mining of a block. This way, a mining pool get split into a few trusted nodes only.

