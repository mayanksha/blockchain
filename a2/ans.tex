%\documentclass[12pt, a3paper]{extarticle}
\documentclass[12pt, a4paper]{article}
\usepackage{graphicx, rotating, geometry}
\graphicspath{{./}}
\usepackage[utf8]{inputenc}
\usepackage{cite}
\usepackage{amsmath}
\usepackage{amsfonts}
\usepackage{mathtools}
\DeclarePairedDelimiter{\ceil}{\lceil}{\rceil}
\newcommand{\bo}{\textbf}
\newcommand{\ti}{\textit}
\newcommand{\ul}{\underline}
\title{Blockchain Assignment 2}
%\subtitle{Assignment 1}
\author{Mayank Sharma, 160392}
\date{February 7 2019}

\begin{document}
%\newcommand{\b}{\textbf}
\maketitle
\section{Q}

\section{Beyond Binary Merkle Trees}
\subsection{}
Alice takes $T_1, T_2$ and $ T_3$ and hashes them firstly so as to obtain $H(T_1), H(T_2)$ and $ H(T_3)$, respectively. She does similar for the rest of the elements too, so as to obtain $H(T_i), i \in {1, 2, ..., 9}$.
\\ \\
Now, she clumps $H(T_1), H(T_2)$ and $ H(T_3)$ to form a new node which we call and then calculates a new hash from these three values which is $H(T_1 \mid T_2 \mid T_3)$.  \\ \\
Similarly, she gets $H(T_4 \mid T_5 \mid T_6)$ and $H(T_7 \mid T_8 \mid T_9)$, and she makes a node of these values. For the root node of k-ary Merkle Tree, 
she hashes the child nodes to get $H(H(T_1 \mid T_2 \mid T_3) \mid H(T_4 \mid T_5 \mid T_6) \mid H(T_7 \mid T_8 \mid T_9))$. This is how she commits to set $S$. \\ \\ 
Suppose at a later stage Bob wants to check if $T_4$ is in $S$. So, she must have handed out a \ti{verification token} to Bob which contained the values 
$H(T_1 \mid T_2 \mid T_3)$ and $H(T_5), H(T_6)$ and $H(T_7 \mid T_8 \mid T_9)$. Alice finds $H(T_4)$ from $T_4$, then finds $H(T_4 \mid T_5 \mid T_6)$, and then finally computes $H(H(T_1 \mid T_2 \mid T_3) \mid H(T_4 \mid T_5 \mid T_6) \mid H(T_7 \mid T_8 \mid T_9))$. If this value matches the earlier calculated hash value of root node of this merkle tree, it's confirmed that $T_4 \in S$ .
\\ 
This is because a root node of Merkle Tree indirectly stores the information about all the elements of set $S$ in the form of Hash Values, and at max we perform $\ceil{\log n}$ hashes.

\subsection{}
For finding the Merkle Tree Root node hash, we have to go all the way from leaf node to the root of tree performing hashes. For a $k$-ary tree, we have to go through the height of tree. Hence, the whole operation takes $O(\log_k (n))$ hashes which is the length of the proof.
\subsection{}
\bo{DOUBT!}
For a binary Merkle Tree, we'll need $\log_2(n)$ hashes. The overhead we'll have for a k-ary tree is 
\begin{align*}
	&= \frac{\log_k (n)}{\log_2 (n)}  \\
	&= \log_2 (k)
\end{align}
The only advantage we'll have while using a k-ary tree is that  (Smaller or Larger Verification Token?)

\section{Hiding vs Binding Components}
Phone numbers are just 10 digits which corresponds to about 26 bits of \bo{valid} phone numbers. Now, Alice can create his phonebook to contain all the permutations of these valid phone numbers and she'll be able to know which all users are using BobCrypt app. \\ \\
This way, she can know which all persons use the BobCrypt app.

\section{Bitcoin Script}
\subsection{}
ScriptSig :
\begin{align*}
	<\text{Password of Alice}>
\end{align*}
\subsection{}
	This method is not at all secure because scriptSig comes in the input field of every transaction, meaning that it is publicly visible to everyone. Anyone who knows which transaction (this is also public) Alice is using to store her password can easily see her password in plaintext.
\subsection{}
Yes, a Pay-to-script-hash fixes the security issue although only temporarily. A P2SH Script is hidden from the sender as well as the Blockchain. Only the person who owns this `redeemScript` and subsequently its hash can use this script to perform any operation.  \\  \\
So, here Alice can create a P2SH script for herself which saves her password in plaintext and she sends the hash of this script. She is the only one who owns this script so no worries of leaking password. When a sender processes such a P2SH transaction, he just pays to this script. But, for some other transaction, this script will act as SigScript and it will be revealed to the whole world when a redeemScript has been used for once. Alice will then have to change her password again. 

\section{Lightweight Clients}
\subsection{}

\section{Bitcoin Lotto}
\subsection{}
To delay the redemption of lottery to Saturday, we can use locktime with an appropriate value. The value that locktime takes is that of a block height, so that once that blockheight has been exceeded, then only a miner can include this transaction in the next block. 
\subsection{}
To carry over the prize of week $n$ over to week $n+1$, we can proceed as follows:
\begin{itemize}
		\item When we are creating a new wallet with Prize Amount for week $n$, we will generate another public-private key pair for a new account for week $n+1$.
		\item At the time of printing the lotteries, we can create a new type of transaction which uses locktime, such that the sender address is the account of week $n$ and the receiver is the week $n+1$ account.
		\item We put a locktime on this transaction such that the locktime corresponds to time 1-2 hr before when we are starting the lottery for week $n+1$.
\end{itemize}
Now, when private key of week $n$ account gets lost, all the funds from week $n$ account will automatically get transferred to week $n+1$ account just before the new week's lottery begins. This ensures that someone else doesn't have the chance to reclaim the previous week's ($n$) and this week's ($n+1$) lottery both at the same time.















\end{document}









